\documentclass[11pt]{article}
\usepackage{amsmath,amssymb,amsthm,epsf,
graphics,enumerate,fancyhdr,marvosym,cancel}
\usepackage[letterpaper, margin=0.9in]{geometry}
\usepackage[scr=boondoxo]{mathalpha}

% ordered lists have letters instead of numbers
\renewcommand\theenumi{\alph{enumi}}
\renewcommand\labelenumi{(\theenumi)}

% subsections have letters instead of numbers
\renewcommand{\thesubsection}{\thesection (\alph{subsection})}

% use hyperref for links but don't draw the ugly boxes around links
\usepackage[hidelinks]{hyperref}

% command to create underlined link to outside websites
\newcommand{\hrefunderline}[2]{\underline{\href{#1}{#2}}}

% font to use for collections
\newcommand{\col}[1]{\mathscr{#1}}
% font to use for random variables
\newcommand{\rv}[1]{\mathsf{#1}}
% the probability metric
\newcommand{\p}{\mathbb{P}}
% expectation value
\newcommand{\ex}{\mathbb{E}\,}
% characteristic functions
\newcommand{\charf}[1]{\mathbf{1}_{#1}}
% the Borell sigma algebra on the reals
\newcommand{\bor}{\col{B}}
% power sets
\newcommand{\pset}[1]{P\left(#1\right)}
% command to emphasize the name of something defined
\newcommand{\defname}[1]{\underline{#1}}
% almost sure convergence
\newcommand{\asto}{\xrightarrow{\text{a.s.}}}
% convergence in probability
\newcommand{\pto}{\xrightarrow{\p}}
% convergence in distribution
\newcommand{\disto}{\xrightarrow{\text{dist}}}
% convergence in expectation
\newcommand{\exto}{\xrightarrow{L^1}}


% symbols for common sets
\newcommand{\ZZ}{\mathbb{Z}}
\newcommand{\NN}{\mathbb{N}}
\newcommand{\FF}{\mathbb{F}}
\newcommand{\RR}{\mathbb{R}}
\newcommand{\QQ}{\mathbb{Q}}
\newcommand{\CC}{\mathbb{C}}

% for congruences
\newcommand{\mmod}[1]{\;(\operatorname{mod} {#1})}

% new theorem style with boldface title and italic content
\newtheoremstyle{step}%            % Name
  {}%                                     % Space above
  {}%                                     % Space below
  {\itshape}%                           % Body font
  {}%                                     % Indent amount
  {\itshape}%                          % Theorem head font
  {:}%                                    % Punctuation after theorem head
  { }%                                    % Space after theorem head, ' ', or \newline
  {}%                                     % Theorem head spec (can be left empty, meaning `normal')

% new theorem style for a gap in the notes
\newtheoremstyle{gap}%            % Name
  {}%                                     % Space above
  {}%                                     % Space below
  {\itshape}%                           % Body font
  {}%                                     % Indent amount
  {\itshape}%                          % Theorem head font
  {!!!}%                                    % Punctuation after theorem head
  {    }%                                    % Space after theorem head, ' ', or \newline
  {}%                                     % Theorem head spec (can be left empty, meaning `normal')

\theoremstyle{theorem}
\newtheorem{theorem}{Theorem}[section]
\newtheorem{lemma}[theorem]{Lemma}
\newtheorem{corollary}[theorem]{Corollary}

\theoremstyle{definition}
\newtheorem{definition}[theorem]{Definition}
\newtheorem{notation}[theorem]{Notation}

\theoremstyle{remark}
\newtheorem{example}[theorem]{Example}
\newtheorem*{remark}{Remark}

\theoremstyle{step}
\newtheorem{step}{Step}[subsection]
\renewcommand{\thestep}{\arabic{step}}

\theoremstyle{gap}
\newtheorem*{gap}{Gap}

\begin{document}

\title{Probability Notes}
\author{Duncan MacIntyre}
\date{\today}
\maketitle
\tableofcontents
\bigskip
\newpage

\section{Some measure theory}

\begin{gap}
The discussion of topologies should probably be removed.
\end{gap}


\begin{definition}
Let \(\Omega\) be a set. A \defname{topology} on \(\Omega\) is a collection \(\col{A} \subset \pset{\Omega}\) that is closed under unions and finite intersections with \(\Omega, \emptyset \in \col{A}\). For \(E \subset \Omega\), if \(E \in \col{A}\), we call \(E\) open, and if \(E^C \in \col{A}\), we call \(E\) closed.
\end{definition}

For example, if \((X, d)\) is a metric space, \(\col{A} = \left\{E | E = \cup_i N_{r_i} \left(x_i\right)\right\}\) is a topology on \(X\).

\begin{gap} The example should be proven.\end{gap}

\begin{definition}
Let \(\Omega\) be a set and \(\col{S} \subset \pset{\Omega}\). Then the \defname{topology generated by \(\col{S}\)} is
\[\col{A} = \left\{E \subset \Omega : E \text{ is a union of finite intersections of sets in } \col{S}.\right\}\]
\end{definition}

\begin{gap} The definition should be more explicit.\end{gap}

\begin{gap} Examples and Furstenberg's theorem should be added here \end{gap}

\begin{definition}
Let \(\Omega\) be a set. An \defname{algebra} is a collection \(\col{A} \subset \pset{\Omega}\) that is closed under finite unions and compliments with \(\Omega \in \col{A}\). If an algebra is also closed under countable unions, we call it a \defname{\(\sigma\)-algebra}.
\end{definition}

\begin{theorem}
Algebras are closed under finite intersections and \(\sigma\)-algebras are closed under countable intersections.
\end{theorem}

\begin{proof}
Let \(\col{A}\) be an algebra and let \(A_1, \ldots, A_n \in \col{A}\). Then \(A_1^C, \ldots, A_n^C \in \col{A}\), so \(\left(\cap_{i=1}^n A_i\right)^C = \cup_{i=1}^n A_i^C \in \col{A}\), so \(\cap_{i=1}^n A_i \in \col{A}\). The proof for \(\sigma\)-algebras is similar.
\end{proof}

\begin{definition}
Let \(\col{f}\) be a \(\sigma\)-algebra on a set \(\Omega\). A \defname{measure} on \(\col{f}\) is a function \(\mu:\col{f} \to [0, \infty]\) such that \(\mu(\emptyset) = 0\) and for all disjoint \(\left(A_i\right)_{i\in \NN} \in \col{f}\) we have \(\mu\left(\cup_i A_i\right) = \sum_i \mu\left(A_i\right)\). We call \((\Omega, \col{f}, \mu)\) a \defname{measure space}.
\end{definition}

\begin{gap}
Add examples of measures and probability measures.
\end{gap}

\begin{theorem}\label{thm.intersection-algebras}
An intersection of \(\sigma\)-algebras is also a \(\sigma\)-algebra.
\end{theorem}

\begin{proof}
Let \((\col{A}_i)\) be uncountably many \(\sigma\)-algebras. Let \(A \in \cap \col{A}_i\). Then \(A^C \in \col{A}_i\) for all \(\col{A}_i\), so \(A^C \in \cap \col{A}_i\). Similarly, let \((A_j)_{j \in \NN} \in \cap \col{A}_i\). Then \(\cup_j A_j \in \col{A_i}\) for all \(\col{A}_i\), so \(\cup_j A_j \in \cap \col{A_i}\).
\end{proof}

Theorem \ref{thm.intersection-algebras} motivates the following definition.

\begin{definition}
Let \(\col{A} \subset \Omega\). The \defname{\(\sigma\)-algebra generated by \(\col{A}\)} is the intersection of all \(\sigma\)-algebras that contain \(\col{A}\). We denote it by \(\sigma(\col{A})\).
\end{definition}

We can think of \(\sigma(\col{A})\) as the smallest \(\sigma\)-algebra containing \(\col{A}\).

\begin{notation}
A Borel \(\sigma\)-algebra is a \(\sigma\)-algebra generated by a topology. We will use \(\bor\) to denote the Borel \(\sigma\)-algebra of \(\RR\) that is generated by the usual open set topology.
\end{notation}

\begin{gap} This section needs to be continued with product spaces, the upper/lower continuity of measures, the construction of the Lebesgue measure, the Caratheodory theorem, and the Dynkin uniqueness theorem.\end{gap}

\section{Probability spaces}

\begin{definition}
Let \((\Omega, \col{f}, \p)\) be a measure space. If \(\p(\col{f}) = 1\), we call \(\p\) a \defname{probability measure}, we call \((\Omega, \col{f}, \p)\) a \defname{probability space}, and we call an element of \(\col{f}\) an \defname{event}.
\end{definition}

\begin{notation}
It is common to use a shorthand notation for events. We often write \(\{\text{condition}\}\) to mean \(\{\omega \in \Omega\;:\;\text{condition}\}\) and we often write \(\p(\text{condition})\) to mean \(\p\left(\{\omega \in \Omega\;:\;\text{condition}\}\right)\).
\end{notation}

\begin{theorem}\label{thm.nestedconvergence}
Let \(\{A_n\}_{n \in \NN}\) be a sequence of events.
\begin{enumerate}
\item
If \(A_1 \subset A_2 \subset A_3 \subset \cdots\) then \(\lim_{n \to \infty} \p(A_n) = \p\left(\cup_{n=1}^\infty A_n\right)\).
\item
If \(A_1 \supset A_2 \supset A_3 \supset \cdots\) then \(\lim_{n \to \infty} \p(A_n) = \p\left(\cap_{n=1}^\infty A_n\right)\).
\end{enumerate}
\end{theorem}

\begin{proof}\ 
\begin{enumerate}
\item
Let \(A = \cup_{n=1}^\infty A_n\). Let \(B_1 = A_1\) and for \(n > 1\) let \(B_n = A_n \setminus A_{n-1}\). Then the \(B_n\) are disjoint.
Now, \(A_n = \cup_{i=1}^n B_i\), so \(\p(A_n) = \sum_{i=1}^n \p(B_n)\). Also \(A = \cup_{n=1}^\infty B_n\) so \(\p(A) = \sum_{i=1}^\infty \p(B_i)\). This sum converges because the partial sums \(\sum_{i=1}^n \p(B_i)\) are monotonically increasing and bounded above by \(1\). The remainder \(R_n = \sum_{i = n+1}^\infty \p(B_n)\) has \(R_n \to 0\) as \(n \to \infty\).
Finally, we see that \(\p(A) - \p(A_n) = R_n \to 0\) as \(n \to \infty\). Thus \(\p(A_n) \to \p(A)\) as \(n \to \infty\).

\item
This follows from the previous part by taking compliments.
\end{enumerate}
\end{proof}

\begin{definition}
If \((A_i)_{i \in \NN}\) are a sequence of events, ``\(A_i\) occurring \defname{infinitely often}'' is the event
\[\{A_i \text{ i.o.}\} = \bigcap_{m \in \NN} \;\bigcup_{n \geq m} A_n\]
and ``\(A_i\) occurring \defname{eventually}'' is the event
\[\{A_i \text{ eventually}\} = \bigcup_{m \in \NN} \;\bigcap_{n \geq m} A_n.\]
\end{definition}

\begin{remark}
\(\{A_i \text{ i.o.}\}\) is the set of elements \(\omega \in \Omega\) that belong in infinitely many of the events \((A_i)\) and \(\{A_i \text{ eventually}\}\) is the set of elements \(\omega \in \Omega\) that belong in all \((A_i)\) after a certain point. We note that \(\{A_i \text{ eventually}\} \subset \{A_i \text{ i.o.}\}\).
\end{remark}

\begin{definition}
The \defname{characteristic function} of an event \(A\) in \(\Omega\) is the function \(\charf{A}:\Omega \to \{0, 1\}\) that has \(\charf{A}(t) = 1\) if \(t \in A\) and \(\charf{A}(t) = 0\) if \(t \not\in A\).
\end{definition}

\begin{theorem}\label{thm.eventsio.charf}
Let \((A_i)_{i \in \NN}\) be a sequence of events.
\begin{enumerate}
\item
\(\charf{\{A_i \text{ i.o.}\}} = \limsup_{i \to \infty} \charf{A_i}\).
\item
\(\charf{\{A_i \text{ eventually}\}} = \liminf_{i \to \infty} \charf{A_i}\).
\end{enumerate}
\end{theorem}

\begin{proof}\ 
\begin{enumerate}
\item
Suppose \(\charf{\{A_i \text{ i.o.}\}}(t) = 1\). Then \(t \in \{A_i \text{ i.o.}\}\), so \(t\) is in infinitely many \(A_i\); call these \(A_{\alpha_i}\). Then \(\charf{A_{\alpha_i}}(t) = 1\) for all \(i\), so \(\lim_{i \to \infty} \charf{A_{\alpha_i}}(t) = 1\) and therefore \(\limsup_{i \to \infty} \charf{A_{i}}(t) = 1\).

Now suppose instead \(\limsup_{i \to \infty} \charf{A_{i}}(t) = 1\). Then there exists a subsequence \(\{A_{\alpha_i}\}\) of \(\{A_i\}\) and an \(N \in \NN\) so that \(\left|\charf{A_{\alpha_i}}(t) - 1 \right| < \frac{1}{2}\) for all \(i \geq N\). Then for these \(i\) it must be that  \(\charf{A_{\alpha_i}}(t) = 1\) so \(t \in A_{\alpha_i}\). Thus \(t\) is in infinitely many \(A_i\) so \(t \in \{A_i \text{ i.o.}\}\) and \(\charf{\{A_i \text{ i.o.}\}}(t) = 1\).

\item
Suppose \(\charf{\{A_i \text{ eventually}\}}(t) = 1\). Then \(t \in \{A_i \text{ eventually}\}\), so there exists an \(N \in \NN\) such that \(t\) is in all \(A_i\) for \(i \geq N\). For these \(i\) we have \(\charf{A_i}(t)=1\), so \(\lim_{i \to \infty} \charf{A_i}(t) = 1\). In particular \(\liminf_{i \to \infty} \charf{A_i}(t) = 1\).

Now suppose instead \(\liminf_{i \to \infty} \charf{A_i}(t) = 1\). Then there is no subsequence \(\{A_{\alpha_i}\}\) of \(\{A_i\}\) with \(\lim_{i \to \infty}\{A_{\alpha_i}\}(t) =0\), so there are finitely many \(A_i\) that do not contain \(t\). It follows that there exists an \(N \in \NN\) such that for all \(i \geq N\), \(t \in A_i\). Therefore \(t \in \{A_i \text{ eventually}\}\) and \(\charf{\{A_i \text{ eventually}\}}(t) = 1\).
\end{enumerate}
\end{proof}

\begin{gap}
The above proof could probably be shortened.
\end{gap}

\begin{theorem}
Any events \(A_n\) have \(\p(A_n \text{ i.o.}) \geq \limsup_{n\to\infty} \p(A_n)\).
\end{theorem}

\begin{gap}
Prove the theorem.
\end{gap}

\begin{gap} Add the Borel-Cantelli Lemma and the definition of the Borel \(\sigma\)-algebra on \(\RR\) here.\end{gap}

\begin{definition}
Let \((\Omega, \col{f}, \p)\) be a probability space. Events \((A_i)_{i \in I} \in \col{f}\) are called \defname{independent} if for any finite \(J \subset I\) we have \(\p\left(\cap_{j \in J} A_j\right) = \prod_{j \in J} \p\left(A_j\right)\).
\end{definition}

\begin{theorem}
If \((A_n)\) are independent and for all \(n\) we define either \(B_n = A_n\) or \(B_n = A_n^C\) then all \((B_n)\) are independent.
\end{theorem}

\begin{gap}
Prove the above theorem.
\end{gap}

\begin{gap}
Add the Goldberg conjecture discussion here, as an example.
\end{gap}

\begin{definition}
Let \((\Omega, \col{f}, \p)\) be a probability space. Collections \((\col{A}_i) \subset \col{f}\) are called \defname{independent} if for any sequence \((A_i)\in\col{f}\) with \(A_i \in \col{A}_i\) the \((A_i)\) are independent.
\end{definition}

\begin{gap}
Add examples of independent and non-independent collections.
\end{gap}


\section{Random variables}

\begin{definition}
Let \((\Omega_1, \col{f}_1, \mu_1)\) and \((\Omega_2, \col{f}_2, \mu_2)\) be measure spaces. A function \(g: \Omega_1 \to \Omega_2\) is called \defname{measurable} if for all \(A \in \col{f}_2\) we have \(g^{-1}(A) \in \col{f}_1\).
\end{definition}

\begin{remark}
This is similar to the topological definition of continuous functions.
\end{remark}


\begin{definition}
A \defname{random variable} on a probability space \((\Omega, \col{f}, \p)\) is a measurable function \(\rv{X}:\Omega \to \RR\), where we use the usual Borel \(\sigma\)-algebra as the \(\sigma\)-algebra on \(\RR\).
\end{definition}

\begin{notation}
When writing down an event involving a random variable \(\rv{X}\), it is common to write \(\rv{X}\) when we really mean \(\rv{X}(\omega)\). For example, we might write \(\p(\rv{X} = 1)\) to mean \(\p\left(\left\{\omega \in \Omega : \rv{X}(\omega) = 1\right\}\right)\).
\end{notation}

\begin{gap}
The notation remark above could be made more clear.
\end{gap}

\begin{gap}
Add some examples of random variables.
\end{gap}

\begin{remark}
We can form an equivalence relation on random variables by writing \(\rv{X} \sim \rv{Y}\) if \(\p(X=Y)=1\), that is, if \(\{\omega \in \Omega : \rv{X}(\omega) \neq \rv{Y}(\omega)\}\) has measure zero. Equivalent random variables have the same probabilistic properties. Some may prefer to think of ``random variables'' as these equivalence classes of functions. In this sense, it doesn't make sense to ask what a random variable's value is for a certain \(\omega \in \Omega\) with \(\p(\{\omega\}) = 0\); it only makes sense to look at the random variable's values on events with measure greater than zero.
\end{remark}

\begin{definition}
The \defname{distribution function} of a random variable \(\rv{X}\) is the function \(F_\rv{X}: \RR \to \RR\) given by \(F(t) = \p(X \leq t)\).
\end{definition}

We immediately see that distribution functions are monotonically increasing and right continuous (i.e., \(\lim_{t\to s^+} F_\rv{X}(t) = F_\rv{X}(s)\) for all \(s \in [0,1)\)). Also, \(F_\rv{X}\) has a jump discontinuity at \(s\) if and only if \(\p(\rv{X} = s) > 0\) (and if so, the jump is of size \(\p(\rv{X} = s)\)). There is an interesting theorem that provides something like a converse.

\begin{theorem}
Let \(F:\RR \to [0,1]\) be monotonically increasing and right continuous (i.e., \(\lim_{t\to s^+} F(t) = F(s)\) for all \(s \in [0,1))\). Then there exists a probability space \((\Omega, \col{f}, \p)\) and a random variable \(\rv{X}:\Omega \to \RR\) such that \(F = F_\rv{X}\).
\end{theorem}

\begin{gap}
Add the two theorem proofs.
\end{gap}

\begin{definition}
Given random variables \((\rv{X}_i)\) let \(\col{A}_i = \{{\rv{X}_i}^{-1}(A) : A \in \bor\}\). We say that \((\rv{X}_i)\) are \defname{independent} if \((\col{A}_i)\) are independent. (Here, \(\bor\) is the usual Borel \(\sigma\)-algebra on \(\RR\).)
\end{definition}

\begin{theorem}
Let \((\rv{X}_i)\) be random variables and let \(\col{A}_i = \left\{\left\{\rv{X}_i \leq t\right\} : t \in \RR\right\}\). Suppose \((\col{A}_i)\) are independent. Then \((\rv{X}_i)\) are independent.
\end{theorem}

\begin{gap}
Prove the theorem.
\end{gap}

\begin{definition}
The \defname{joint distribution function} of random variables \(\rv{X}_1, \ldots, \rv{X}_n\) is the function \(F:\RR^n \to \RR\) given by \[F\left(\langle t_1, \ldots, t_n\rangle\right) = \p \left(\rv{X}_i \leq t_i \;\forall\; i \in \{0, \ldots, n\}\right).\]
\end{definition}

\begin{gap}
Add notes on stochastic domination, percolation, etc.
\end{gap}

\section{Sequences of random variables}

\begin{definition}
Let \(\rv{X}_n, \rv{X} : \Omega\to \RR\) be random variables. We say that \(\{\rv{X}_n\}\) \defname{converges almost surely} to \(\rv{X}\) and write \(\rv{X}_n \asto \rv{X}\) if \(\p\left(\left\{\omega \in \Omega : \rv{X}_n(\omega) \to \rv{X}(\omega)\right\}\right) = 1\). We say that \(\{\rv{X}_n\}\) \defname{converges in probability} to \(\rv{X}\) and write \(\rv{X}_n \pto \rv{X}\) if for all \(\epsilon>0\) we have \(\p\left(\left|\rv{X}_n-\rv{X}\right| < \epsilon\right) \to 1\) as \(n \to \infty\).
\end{definition}

\begin{definition}\label{def.disto}
Let \(\rv{X}_n : \Omega_n \to \RR\) and \(\rv{X}:\Omega \to \RR\) be random variables. We say that \(\{\rv{X}_n\}\) \defname{converges in distribution} to \(\rv{X}\) and write \(\rv{X}_n \disto \rv{X}\) if \(F_{\rv{X}_n}(t) \to F_\rv{X}(t)\) for all \(t\in\RR\) such that \(\p\left(\rv{X}=t\right) = 0\). 
\end{definition}

\begin{theorem}
Let \(\rv{X}_n, \rv{X} : \Omega\to \RR\) be random variables.
\begin{enumerate}
\item
If \(\rv{X}_n \asto \rv{X}\) then \(\rv{X}_n \pto \rv{X}\).
\item
If \(\rv{X}_n \pto \rv{X}\) then \(\rv{X}_n \disto \rv{X}\).
\end{enumerate}
\end{theorem}

\begin{proof}\ 
\begin{enumerate}
\item
Suppose \(\rv{X}_n \asto \rv{X}\). Let \(S = \left\{\omega \in \Omega : \rv{X}_n(\omega) \to \rv{X}(\omega)\right\}\). Then \(\p(S) = 1\). 

Let \(\epsilon > 0\) and let \(A_n^\epsilon = \{|\rv{X}_n-X| < \epsilon\}\). For all \(\omega \in S\), there exists an \(N\) such that \(|\rv{X}_n(\omega) - X(\omega)| < \epsilon\) for all \(n \geq N\); in other words, if \(n \geq N\) then \(\omega \in A_n^\epsilon\). Thus \(S \subset \{A_n^\epsilon \text{ eventually}\}\). Since \(\p(S) = 1\), we must have \(\p\left(\{A_n^\epsilon \text{ eventually}\}\right) = 1\).

...


%
%\(\rv{X}_n \asto \rv{X}\) if and only if \(\p\left(A_n^\epsilon \text{ i.o.}\right) = 1\) for all \(\epsilon > 0\).
%
%\(\rv{X}_n \pto \rv{X}\) if and only if \(\p\left(A_n^\epsilon\right) \to 1\) as \(n \to \infty\) for all \(\epsilon > 0\).


\item
...
\end{enumerate}
\end{proof}

\begin{gap}
Finish the proof. Add counterexamples to show that converses not true.
\end{gap}

\begin{theorem}{(Uniqueness)}
\begin{enumerate}
\item
Suppose \(\rv{X}_n \pto \rv{X}\) and \(\rv{X}_n \pto \rv{Y}\). Then \(\p(\rv{X}=\rv{Y}) = 1\).
\item
Suppose \(\rv{X}_n \disto \rv{X}\) and \(\rv{X}_n \disto \rv{Y}\). Then \(F_\rv{X} = F_\rv{Y}\).
\end{enumerate}
\end{theorem}

\begin{proof}\ 
\begin{enumerate}
\item
Let \(\epsilon > 0\).
We have \(|\rv{X}-\rv{Y}| \leq |\rv{X}_n-\rv{X}| + |\rv{X}_n - \rv{Y}|\), so if \(|\rv{X}-\rv{Y}|\geq \epsilon\) then either \(|\rv{X}_n - \rv{X}| \geq \frac{\epsilon}{2}\) or \(|\rv{X}_n - \rv{Y}| \geq \frac{\epsilon}{2}\). Thus \[\{|\rv{X}-\rv{Y}|\geq \epsilon\} \subset \left\{|\rv{X}_n-\rv{X}| \geq \frac{\epsilon}{2}\right\} \cup \left\{|\rv{X}_n - \rv{Y}| \geq \frac{\epsilon}{2}\right\}\]
so
\[\p(|\rv{X}-\rv{Y}|\geq \epsilon) \leq \p\left(|\rv{X}_n-\rv{X}| \geq \frac{\epsilon}{2}\right) + \p\left(|\rv{X}_n - \rv{Y}| \geq \frac{\epsilon}{2}\right) \to 0\]
as \(n \to \infty\). Therefore \(\p(|\rv{X}-\rv{Y}| < \epsilon) = 1\).

Take \(A_1 = \{|\rv{X}-\rv{Y}| < 1\}\) and for \(n \geq 2\) take \(A_n = A_{n-1} \cap \{|\rv{X}-\rv{Y}| < \frac{1}{n}\}\). Then all \(\p(A_n) = 1\) and \(A_1 \supset A_2 \supset A_3 \supset \cdots\). Let \(A = \cap_n A_n\). Then by Theorem \ref{thm.nestedconvergence}, \(\p(A) = 1\). Also for all \(\omega \in A\) we must have \(\rv{X}(\omega) = \rv{Y}(\omega)\). We conclude that \(\p(\rv{X}=\rv{Y}) = 1\).

\item
From Definition \ref{def.disto} it is immediately clear that \(F_\rv{X} = F_\rv{Y}\) except where \(F_\rv{X}\) is discontinuous. This can happen for at most countably many \(t \in \RR\) (because at each discontinuity there exists a unique rational not in the range of \(F_\rv{X}\)). Then \(\{s \in \RR:F_\rv{X}(s) = F_\rv{Y}(s)\}\) is dense everywhere in \(\RR\). Thus for any \(t\in\RR\) where \(F_\rv{X}\) is discontinuous, there exist \(t_n\in \RR\) with \(t_n \to t\) and \(t_n > t\) and \(F_\rv{X}(t_n) = F_\rv{Y}(t_n)\) for all \(n\). Recalling that \(F_\rv{X}\) and \(F_\rv{Y}\) are right-continuous, we must have \(F_\rv{X}(t) = \lim_{n \to \infty} F_\rv{X}(t_n) = \lim_{n \to \infty} F_\rv{Y}(t_n) = F_\rv{Y}(t) \). This shows that \(F_\rv{X} = F_\rv{Y}\) everywhere.
\end{enumerate}
\end{proof}

\begin{theorem}
Suppose \(\rv{X}_n \pto \rv{X}\). Then there exists a subsequence \(\{\rv{X}_{n_k}\}\) of \(\{\rv{X}_n\}\) such that \(\rv{X}_{n_k} \asto \rv{X}\).
\end{theorem}

\begin{gap}
Prove the theorem.
\end{gap}


\begin{theorem}{(Skorohod's Representation Theorem)}\ 

Suppose \(\rv{X}_n \disto \rv{X}\). Then there exist \(\rv{Y}_n, \rv{Y}\) such that all \(F_{\rv{X}_n} = F_{\rv{Y}_n}\) and \(F_\rv{X} = F_\rv{Y}\) and \(\rv{Y}_n \asto \rv{Y}\).
\end{theorem}

\begin{gap}
Prove the theorem.
\end{gap}


\section{Expectation values}

\begin{definition}
A \defname{simple random variable} is a random variable \(\rv{X}:\Omega \to \RR\) such that \(\rv{X} = \sum_{i=1}^\infty y_i \charf{A_i}\) for some \(y_i \in \RR\) and disjoint \(A_i \subset \Omega\).
\end{definition}

\begin{theorem}
For any random variable \(\rv{X}\) there exist simple random variables \(\rv{X}_n\) such that for all \(n\) we have \(\rv{X}_n \leq \rv{X}_{n+1}\) everywhere and \(\rv{X}_n \to \rv{X}\).
\end{theorem}

\begin{definition}
Let \(\rv{X} = \sum_{i=1}^\infty y_i \charf{A_i}\) be a simple random variable (with disjoint \(A_i\)). The \defname{expectation} of \(\rv{X}\) is \(\ex \rv{X} = \sum_{i=1}^\infty y_i \p(A_i)\).
\end{definition}

\begin{theorem}\ 
\begin{enumerate}
\item If \(\rv{X} \geq 0\) everywhere then \(\ex \rv{X} \geq 0\).
\item For simple random variables \(\rv{X}\) and \(\rv{Y}\) and \(a \in \RR\) we have \(\ex a \rv{X} = a \ex \rv{X}\) and \(\ex (X+Y) = \ex X + \ex Y\) (linearity).
\item If \(\rv{X} \leq \rv{Y}\) everywhere then \(\ex\rv{X} \leq \ex\rv{Y}\) (monotonicity).
\end{enumerate}
\end{theorem}

\begin{gap}
Prove the theorem.
\end{gap}

Now that we have defined the expectation for simple random variables, we want to extend the definition to all random variables. First we extend it to positive random variables.

\begin{definition}
Let \(\rv{X}: \Omega \to \RR\) be a random variable with \(\rv{X} > 0\) everywhere. Define the \defname{expectation} of \(\rv{X}\) to be
\[\ex \rv{X} = \sup \left\{\ex \rv{Y} : \rv{Y} \text{ is a simple random variable with } \rv{Y} \leq \rv{X} \text{ everywhere} \right\}.\]
\end{definition}

\begin{theorem}{(Monotone Convergence Theorem)}\ 

Suppose \(\rv{X}_n \to \rv{X}\) and always \(\rv{X}_{n+1} \geq \rv{X}_{n}\) everywhere. Then \(\ex \rv{X}_n \to \ex \rv{X}\).
\end{theorem}

\begin{gap}
Prove the theorem.
\end{gap}

\begin{gap}
Discuss the expectation for random variables in general, not just non-negative random variables.
\end{gap}

\begin{notation}
Instead of \(\ex \rv{X}\) we sometimes write \(\int \rv{X} d\p\) or \(\int \rv{X}(\omega) d\p(\omega)\).
\end{notation}

\begin{definition}
We say that \(\rv{X}_n\) \defname{converge in expectation} to \(\rv{X}\) and write \(\rv{X}_n \exto \rv{X}\) if \(\ex |\rv{X}_n - \rv{X}| \to 0\) as \(n \to \infty\).
\end{definition}

\begin{theorem}{(Markov's Inequality)}
If \(\rv{X} > 0\) everywhere then \(\p(\rv{X} \geq t) \leq \frac{\ex \rv{X}}{t}\).
\end{theorem}

\begin{gap}
Prove the theorem.
\end{gap}

\begin{theorem}{(Fatou's Lemma)}
If \(\rv{X}_n \geq 0\) everywhere then
\begin{enumerate}
\item
\(\liminf \ex \rv{X}_n \geq \ex \liminf \rv{X}_n\)
\item
\(\limsup \ex \rv{X}_n \leq \ex \limsup \rv{X}_n\).
\end{enumerate}
\end{theorem}

\begin{gap}
Prove the theorem.
\end{gap}

\begin{theorem}{(Dominated Convergence Theorem)}
Let \(\rv{X}_n, \rv{X}, \rv{Y}\) be random variables with all \(|\rv{X}_n| \leq \rv{Y}\) and \(\rv{X}_n \asto \rv{X}\) and \(\ex \rv{Y}\) is finite. Then \(\ex \rv{X} = \lim_{n \to \infty} \ex \rv{X}_n\).
\end{theorem}

\begin{gap}
Prove the theorem.
\end{gap}

\begin{gap}
Add the example theorem in graph theory.
\end{gap}

\begin{theorem}
If \(\rv{X}_i\) are independent random variables then \(\ex \prod_i \rv{X}_i = \prod_i \ex \rv{X}_i\).
\end{theorem}

\begin{gap}
Prove the theorem.
\end{gap}

\begin{theorem}
If \(\rv{X}_i\) are random variables each with finite expectation then \(\ex \prod_i \rv{X}_i = \prod_i \ex \rv{X}_i\).
\end{theorem}

\begin{gap}
Prove the theorem.
\end{gap}

\begin{theorem}
Random variables \(\rv{X}_i\) are independent if and only if for all bounded functions \(f_i\), \(\ex \prod_i f_i(\rv{X}_i) = \prod_i \ex f_i(\rv{X}_i)\).
\end{theorem}

\begin{gap}
Prove the theorem.
\end{gap}

\begin{definition}
Random variables \(\rv{X}\) and \(\rv{y}\) are called \defname{uncorrelated} if \(\ex (\rv{X} \rv{Y}) = (\ex \rv{X})(\ex \rv{Y})\).
\end{definition}

\begin{theorem}{(Law of Large Numbers)}\ 

Let \(\rv{X}_n\) be independent identically-distributed random variables each with \(\ex |\rv{X}_n|\) being finite. Then \(\frac{1}{n} \sum_{i=1}^n \rv{X}_i \asto \ex \rv{X}_1\) as \(n \to \infty\).
\end{theorem}




















\end{document}
